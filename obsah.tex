%=========================================================================
% (c) Michal Bidlo, Bohuslav Křena, 2008

\chapter{Úvod}
Počítačové sítě, zejména Internet, v dnešním světě zaujímají jednu z nejvýznamnějších rolí. Počínaje výzkumem a vědeckými experimenty, konče běžným životem většiny lidí. Jen za posledních deset let se počet uživatelů Internetu více než ztrojnásobil z cca jedné miliardy lidí na tři miliardy. Počítačové sítě propujují celý svět a jsou neustále rozšiřovány, vylepšovány a modernizovány. To vede k větším nárokům na použité technologie a zdroje. 

Avšak se zvyšujícím počtem uživatelů roste i počet útoků na různé počítačové sítě, kterými se útočníci snaží získat informace či poškodit oběť. Síťový útok\cite{rfcAttack} je definován jako záměrný akt, kde se entita snaží překonat bezpečnostní služby a porušit bezpečnost systému. Vznikají tím pádem systémy na detekci takovýchto útoků, aby správcí sítí dokázali reagovat na vzniklou situaci.

Jeden z těchto systémů vznikl ve sdružení CESNET s názvem Nemea. Tento framework analyzuje síťový provoz a zaznamenává podezřelé toky jako agregované události do databáze. Na větší síti (stovky až tisíce připojených zařízení) je takovýchto událostí vytvořeno až několik tisíc denně. S tím nastává problém jak dané události jednoduše analyzovat a rozpoznat na jaké události se zaměřit a na které nebrát zřetel.

Cílem této bakalářské práce je vytvořit aplikaci pro vizuální analýzu bezpečnostních událostí na síti monitorované s pomocí frameworku Nemea, tak aby správce sítě dokázal rychle a jednoduše rozpoznat významný útok na síť. Důležitým aspektem vytvořené aplikace je důraz na použití moderních knihoven podporující tvorbu dynamických webových aplikací, které jsou dostupné na různých typech zařízeních. Společně s tím je kladen důraz na uživatelskou přívětivost a jednoduchost prostředí, ve kterém bude probíhat vizuální analýza událostí.

Aplikace bude pracovat s konkrétním formátem dat nazvaný IDEA. Tento formát dat je specifikován sdružením CESNET a slouží jako prostředek pro sdílení dat bezpečnostních událostí mezi různými systémy. Díky tomu lze systém kdykoliv přenést na jiný zdroj databáze než je systém Nemea, např. v rámci sdružení CESNET na systém Warden.

Celou aplikaci navíc bude možno libovolně přizpůsobit tak, aby vyhovovala potřebám daného správce sítě. V aplikaci bude zavedena i technika zvaná {\it drill-down}, která napomáhá rychlé a přehledné analýze velkého množství dat bez ztráty informací o analyzované události.

Aplikace bude integrována do současného Nemea frameworku pod názvem Nemea Dashboard a bude s ním společně distribuována jako front~end celého systému.

\break
\chapter{Monitoring síťových dat}
\section{Nemea}
\subsection{Hlavní komponenty}
\subsubsection{libtrap}
\subsubsection{UniRec}
\section{IDEA}
\section{Další monitorovací systémy}

\chapter{Technologie}
\section{Dostupné technologie}
\section{Výběr technologií}
\section{Zvolené technologie}

\chapter{Architektura aplikace}
\section{REST API}
\section{Databáze událostí}
\section{GUI}

\chapter{Implementace}
\section{Backend}
\section{Frontend}
\section{Zabezpečení}
\section{Distribuce}

\chapter{Dosažené výsledky}
\section{Názory uživatelů}
\section{Nasazení v praxi}

\chapter{Závěr}
Závěrečná kapitola obsahuje zhodnocení dosažených výsledků se zvlášť vyznačeným vlastním přínosem studenta. Povinně se zde objeví i zhodnocení z pohledu dalšího vývoje projektu, student uvede náměty vycházející ze zkušeností s řešeným projektem a uvede rovněž návaznosti na právě dokončené projekty.

%=========================================================================
